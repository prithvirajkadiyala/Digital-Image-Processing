% !TEX TS-program = pdflatex
% !TEX encoding = UTF-8 Unicode

% This is a simple template for a LaTeX document using the "article" class.
% See "book", "report", "letter" for other types of document.

\documentclass[11pt]{article} % use larger type; default would be 10pt

\usepackage[utf8]{inputenc} % set input encoding (not needed with XeLaTeX)

%%% Examples of Article customizations
% These packages are optional, depending whether you want the features they provide.
% See the LaTeX Companion or other references for full information.

%%% PAGE DIMENSIONS
\usepackage{geometry} % to change the page dimensions
\geometry{a4paper} % or letterpaper (US) or a5paper or....
% \geometry{margin=2in} % for example, change the margins to 2 inches all round
% \geometry{landscape} % set up the page for landscape
%   read geometry.pdf for detailed page layout information

\usepackage{graphicx} % support the \includegraphics command and options

% \usepackage[parfill]{parskip} % Activate to begin paragraphs with an empty line rather than an indent

%%% PACKAGES
\usepackage{booktabs} % for much better looking tables
\usepackage{array} % for better arrays (eg matrices) in maths
\usepackage{paralist} % very flexible & customisable lists (eg. enumerate/itemize, etc.)
\usepackage{verbatim} % adds environment for commenting out blocks of text & for better verbatim
\usepackage{subfig} % make it possible to include more than one captioned figure/table in a single float
\usepackage{graphicx} % includes image inserting into PDFs 
% These packages are all incorporated in the memoir class to one degree or another...

%%% HEADERS & FOOTERS
\usepackage{fancyhdr} % This should be set AFTER setting up the page geometry
\pagestyle{fancy} % options: empty , plain , fancy
\renewcommand{\headrulewidth}{0pt} % customise the layout...
\lhead{}\chead{}\rhead{}
\lfoot{}\cfoot{\thepage}\rfoot{}

%%% SECTION TITLE APPEARANCE
\usepackage{sectsty}
\allsectionsfont{\sffamily\mdseries\upshape} % (See the fntguide.pdf for font help)
% (This matches ConTeXt defaults)

%%% ToC (table of contents) APPEARANCE
\usepackage[nottoc,notlof,notlot]{tocbibind} % Put the bibliography in the ToC
\usepackage[titles,subfigure]{tocloft} % Alter the style of the Table of Contents
\renewcommand{\cftsecfont}{\rmfamily\mdseries\upshape}
\renewcommand{\cftsecpagefont}{\rmfamily\mdseries\upshape} % No bold!


%%%Matlab Code Requirements
\usepackage{listings}
\usepackage{color} %red, green, blue, yellow, cyan, magenta, black, white
\definecolor{mygreen}{RGB}{28,172,0} % color values Red, Green, Blue
\definecolor{mylilas}{RGB}{170,55,241}
\graphicspath{ {images/} }
%%% END Article customizations

%%% The "real" document content comes below...

\title{Homework 6}
\author{Prithviraj Kadiyala}
%\date{} % Activate to display a given date or no date (if empty),
         % otherwise the current date is printed 

\begin{document}
\maketitle

\section{First Answer}

\lstset{language=Matlab,%
    %basicstyle=\color{red},
    breaklines=true,%
    morekeywords={matlab2tikz},
    keywordstyle=\color{blue},%
    morekeywords=[2]{1}, keywordstyle=[2]{\color{black}},
    identifierstyle=\color{black},%
    stringstyle=\color{mylilas},
    commentstyle=\color{mygreen},%
    showstringspaces=false,%without this there will be a symbol in the places where there is a space
    numbers=left,%
    numberstyle={\tiny \color{black}},% size of the numbers
    numbersep=9pt, % this defines how far the numbers are from the text
    emph=[1]{for,end,break},emphstyle=[1]\color{red}, %some words to emphasise
    %emph=[2]{word1,word2}, emphstyle=[2]{style},    
}
\subsection*{Matlab Code}
\lstinputlisting{HW06_1.m}

Discussion:\break
\break
\begin{flushleft}
\textbf{Results:} The Camera99.bin Image contains salt and pepper noise. \break


\textbf{Median Filter:} After applying the median filter we can see that the salt and pepper noise is
removed from the image but the image appears to be little bit noisy i.e. the image is not as clear
and lost some details. Although, there is some loss in details of the image, the image is
acceptable as the salt and pepper noise is removed completely.\break


\textbf{Open:} The result of applying the open operation to camera99.bin can be seen. After applying the
open operation we can see that white spots are removed from the image, but
still there are black spots on the image. \break


\textbf{Close:} The result of applying the close operation to camera99.bin can be seen. After applying the
operation we can see that the result is complementary as to that of Open. In close operation the
white spots are preserved and the black spots are removed from the image.

\end{flushleft}

\begin{figure}
 \centering
	\includegraphics{1aa.png}
	\includegraphics{1ab.png}
\end{figure}
\begin{figure}
 \centering
	\includegraphics{1ac.png}
	\includegraphics{1ad.png}
\end{figure}

\clearpage

\section {Second Answer}
\subsection*{Matlab Code}
\lstinputlisting{HW06_2.m}
Discussion:\break
\break

\begin{flushleft}
\textbf{Results:} The Camera9.bin Image contains salt and pepper noise \break


\textbf{Median Filter:}After applying the median filter we can see that the salt and pepper noise is
removed from the image but the image appears to be little bit noisy i.e. the image is not as clear
and lost some details. Although, there is some loss in details of the image, the image is
acceptable as the salt and pepper noise is removed completely.\break


\textbf{Open:}The result of applying the open operation to camera9.bin can be seen. After applying the
open operation we can see that white spots (positive spikes) are removed from the image, but
still there are black spots on the image. \break


\textbf{Close:} The result of applying the close operation to camera9.bin can be seen. After applying the
operation we can see that the result is complementary as to that of Open. In close operation the
white spots are preserved and the black spots (negative spikes) are removed from the image.

\end{flushleft}


\begin{figure}
 \centering
	\includegraphics{2aa.png}
	\includegraphics{2ab.png}
\end{figure}
\begin{figure}
 \centering
	\includegraphics{2ac.png}
	\includegraphics{2ad.png}
\end{figure}



\end{document}
